\documentclass[10pt,titlepage]{article}
\usepackage{graphicx}
%\usepackage{graphics}
\usepackage{epsfig}
\usepackage{amsmath}
\usepackage{amssymb}
\usepackage{amsthm}
\usepackage{booktabs}
\usepackage{stmaryrd}
\usepackage{url}
%\usepackage{longtable}
\usepackage[figuresright]{rotating}
\usepackage{titlesec}
\usepackage{hyperref} 
\usepackage{polski}
\usepackage[utf8]{inputenc}
\usepackage[T1]{fontenc}

\usepackage{geometry}
\usepackage{pslatex}
%\usepackage{ulem}

\usepackage{listings}
\usepackage{url}
\usepackage{Here}

\usepackage{color}
\definecolor{szary}{gray}{0.75}% jasnoszary

%\setlength{\textwidth}{400pt}

\lstset{numbers=left,
			numberstyle=\tiny, 
			basicstyle=\scriptsize\ttfamily, 
			breaklines=true, 
			captionpos=b, 
			tabsize=2}

\usepackage[ruled,vlined,linesnumbered]{algorithm2e}

\vfuzz2pt % Don't report over-full v-boxes if over-edge is small
\hfuzz2pt % Don't report over-full h-boxes if over-edge is small


\newcommand{\RR}{\mathbb{R}}
\newcommand{\NN}{\mathbb{N}}
\newcommand{\QQ}{\mathbb{Q}}
\newcommand{\ZZ}{\mathbb{Z}}
\newcommand{\TAB}{\hspace{0.50cm}}
\newcommand{\IFF}{\leftrightarrow}
\newcommand{\IMP}{\rightarrow}

\newcommand{\PRZYKLAD}[1]{\par \noindent{\color{blue}PRZYKŁAD:}\\ {\color{szary}#1}\par}

\newtheorem{theorem}{Twierdzenie}[section]
\newtheorem{lemma}{Lemat}[section]
\newtheorem{example}{Przykład}[section]
\newtheorem{corollary}{Wniosek}[section]
\newtheorem{definition}{Definicja}[section]

\hyphenation{wszy-stkich ko-lu-mnę każ-da od-leg-łość
   dzie-dzi-ny dzie-dzi-na rów-nych rów-ny
   pole-ga zmie-nna pa-ra-met-rów wzo-rem po-cho-dzi
   o-trzy-ma wte-dy wa-run-ko-wych lo-gicz-nie
   skreś-la-na skreś-la-ną cał-ko-wi-tych wzo-rów po-rzą-dek po-rząd-kiem
   przy-kład pod-zbio-rów po-mię-dzy re-pre-zen-to-wa-ne
   rów-no-waż-ne bi-blio-te-kach wy-pro-wa-dza ma-te-ria-łów
   prze-ka-za-nym skoń-czo-nym mo-żesz na-tu-ral-na cią-gu tab-li-cy
   prze-ka-za-nej}

\titleclass{\subsubsubsection}{straight}[\subsection]

\newcounter{subsubsubsection}[subsubsection]
\renewcommand\thesubsubsubsection{\thesubsubsection.\arabic{subsubsubsection}}
\renewcommand\theparagraph{\thesubsubsubsection.\arabic{paragraph}} % optional; useful if paragraphs are to be numbered

\titleformat{\subsubsubsection}
  {\normalfont\normalsize\bfseries}{\thesubsubsubsection}{1em}{}
\titlespacing*{\subsubsubsection}
{0pt}{3.25ex plus 1ex minus .2ex}{1.5ex plus .2ex}

\makeatletter
\renewcommand\paragraph{\@startsection{paragraph}{5}{\z@}%
  {3.25ex \@plus1ex \@minus.2ex}%
  {-1em}%
  {\normalfont\normalsize\bfseries}}
\renewcommand\subparagraph{\@startsection{subparagraph}{6}{\parindent}%
  {3.25ex \@plus1ex \@minus .2ex}%
  {-1em}%
  {\normalfont\normalsize\bfseries}}
\def\toclevel@subsubsubsection{4}
\def\toclevel@paragraph{5}
\def\toclevel@paragraph{6}
\def\l@subsubsubsection{\@dottedtocline{4}{7em}{4em}}
\def\l@paragraph{\@dottedtocline{5}{10em}{5em}}
\def\l@subparagraph{\@dottedtocline{6}{14em}{6em}}
\makeatother

\setcounter{secnumdepth}{4}
\setcounter{tocdepth}{4}

\begin{document}

\pagestyle{empty} %To jest strona tytułowa, bez numeracji

\begin{titlepage}
\vspace*{\fill}
\begin{center}
\begin{picture}(300,510)
  \put( 10,520){\makebox(0,0)[l]{\large \bf \textsc{Wydział Podstawowych Problemów Techniki}}}
  \put( 10,500){\makebox(0,0)[l]{\large \bf \textsc{Politechniki Wrocławskiej}}}
  \put( 70,280){\makebox(0,0)[l]{\Huge  \bf \textsc{Tytuł pracy}}}
  \put(100,240){\makebox(0,0)[l]{\large     \textsc{Imię i nazwisko}}}

  \put(170, 80){\makebox(0,0)[l]{\large  {Praca inżynierska napisana}}}
  \put(170, 60){\makebox(0,0)[l]{\large  {pod kierunkiem}}}
  \put(170, 40){\makebox(0,0)[l]{\large  {..........................}}}

  \put(100,-80){\makebox(0,0)[bl]{\large \bf \textsc{Wrocław 2014}}}
\end{picture}
\end{center}
\vspace*{\fill}
\end{titlepage}

\tableofcontents

\newpage

\pagestyle{headings}  %Zaczynamy właściwą część dokumentu

\section*{Wstęp}      %* oznacza, że ta sekcja nie będzie numerowana   

Praca dyplomowa inżynierska jest dokumentem opisującym zrealizowany system techniczny. 
Praca powinna być napisana poprawnym językiem odzwierciedlającym aspekty techniczne (informatyczne) omawianego zagadnienia. Praca możemy pisać w liczbie mnogiej (np. ,,omówimy'', ,,zdefiniujemy'', \ldots). 
W poniższym dokumencie przykłady sformułowań oznaczono kolorem niebieskim. 
W opisie elementów systemu, komponentów, elementów kodów źródłowych, nazw plików, wejść i wyjść konsoli należy stosować czcionkę stałej szerokości, np: 
{\color{szary}zmienna \verb|wynik| przyjmuje wartość zwracaną przez funkcję \verb|dodaj(a,b)|, 
dla argumentów \verb|a| oraz \verb|b| przekazywanych \ldots}.

Dokument ten jest może być potraktowany jako szablom pracy: jest podzielony na rozdziały zawierające analizę zagadnienia, 
opis projektu systemu oraz implementację. 
  
Wstęp pracy (nie numerowany) powinien składać się z czterech części (które nie są wydzielane jako osobne podrozdziały): 
zakresu pracy, celu, analizy i porównania istniejących rozwiązań oraz przeglądu literatury, opisu zawartości pracy.

Każdy rozdział powinien rozpoczynać się od akapitu wprowadzającego, w którym zostaje w skrócie omówiona zawartość tego rozdziału.

\PRZYKLAD{
\marginpar{Krótkie omówienie celu pracy}
Celem zrealizowanej pracy dyplomowej było zaprojektowanie oraz zbudowanie wielowarstwowego, rozproszonego 
systemu informatyczne typu ,,B2B'', wspierającego wymianę danych pomiędzy przedsiębiorstwami. 
Systemy tego typu, konstruowane dla dużych korporacji, charakteryzują się złożoną 
struktura poziomą i pionową, w której dokumenty \ldots. 

Oto założenia funkcjonalne, które spełniać miała zrealizowana aplikacja:
\begin{itemize}
  \item wspieranie zarządzania obiegiem dokumentów wewnątrz korporacji z uwzględnieniem \ldots,
	\item wspieranie zarządzania zasobami ludzkimi z uwzględnieniem modułów kadrowych oraz \ldots,
	\item gotowość do uzyskania certyfikatu ISO \ldots,
	\item \ldots.
\end{itemize}

\par
\marginpar{Porównanie z podobnymi systemami}
Istnieje szereg aplikacji o zbliżonej funkcjonalności: \ldots, przy czym \ldots.
Zrealizowany projekt, w przeciwieństwie do istniejących komercyjnych rozwiązań, cechuje się bardzo prostym
interfacem użytkownika.\\

\par
Praca składa się z czterech rozdziałów.\marginpar{Struktura pracy}
W rozdziale pierwszym omawiamy strukturę przedsiębiorstwa \ldots, 
charakteryzujemy grupy użytkowników oraz przedstawiamy procedury związane z obiegiem dokumentów. Szczegółowo opisujemy mechanizmy \ldots. Przedstawiamy uwarunkowania prawne udostępniania informacji \ldots, w ramach \ldots.

W rozdziale drugim przedstawiamy szczegółowy projekt systemy w notacji UML. Wykorzystujemy diagramy \ldots.
Opisujemy w pseudokodzie i omawiamy algorytmy generowania \ldots.

W rozdziale trzecim opisujemy technologie implementacji projektu: wybrany język programowania, biblioteki, system zarządzania bazą danych, itp.  Przedstawiamy dokumentację techniczną kodów źródłowych interfejsów poszczególnych modułów systemu. Przedstawiamy sygnatury metod publicznych oraz \ldots.

W rozdziale czwartym przedstawiamy sposób instalacji i wdrożenia systemu w środowisko docelowym.

Końcowy rozdział zawiera  podsumowanie uzyskanych wyników.\\

\par


Projekt zrealizowano w technologii PHP + MySQL. \marginpar{Narzędzia}
Wykorzystano język HTML5 z elementami CSS3 
oraz do obsługi poprawności kilku formularzy skorzystano z języka JavaScipt.
}

Bibliografię organizujemy w postaci pliku BibTex'a. 
Po dodaniu nowego cytowania należy skompilować plik tex'owy, nastepnie skompilować plik bibtex'a (bibtex.exe) i ponownie
skompilować plik tex'owy.  
Oto przykład cytowania z wykorzystaniem tego formatu
(źródło cytowań znajduje się pliku P2P.bib):

\PRZYKLAD{
W pracy \cite{Chord2002} omówiono główne założenia projektu omówionego na stronach \cite{ChordProject}.
W trakcie realizacji systemu wykorzystaliśmy pomysł poprawienia współczynnika jednorodności obciążenia węzłów 
oparty na pomysłach zawartych w pracy \cite{JCIRandom}  
}

Do bibliografi należy dołączyć standardową literaturę oraz literaturę specyficzną dla realizowanego projektu.
Np., jeśli realizowana będzie baza danych w MySQL i obsługiwać ją będziemy w PHP, to dołączmy np. książkę 
\cite{converse2004php5}. 

\par
\textbf{Przed rozpoczęciem redagowania pracy zmień koniecznie zmień nazwę tego pliku. 
Nazwij go np. PD\_XXX\_2014.tex, gdzie XXX to Twoje inicjały.
}
%%%%%%%%%%%%%%%%%%%%%%%%%%%%%%%%%%%%%%%%%%%%%%%%%%%%%%%%%%%%%%%%%%%%%%%%%%%%%%
%%%%%%%%%%%%%%%%%%%%%%%%%%%%%%%%%%%%%%%%%%%%%%%%%%%%%%%%%%%%%%%%%%%%%%%%%%%%%%

\section{Analiza istniejących rozwiązań}

W tym rozdziale należy przedstawić analizę zagadnienia, które podlega informatyzacji. 
Należy zidentyfikować i opisać obiekty składowe rozważanego wycinka rzeczywistości i ich 
wzajemne relacje (np. użytkowników systemu i ich role). 
Należy szczegółowo omówić procesy jakie zachodzą w systemie i które będą informatyzowane, takie jak np. przepływ dokumentów.
Należy sprecyzować i wypunktować założenia funkcjonalne i pozafunkcjonalne dla projektowanego systemu.
Jeśli istnieją aplikacje realizujące dowolny podzbiór zadanych funkcjonalności realizowanego systemu 
należy przeprowadzić ich analizę porównawczą, wskazując na różnice bądź innowacyjne elementy, 
które projektowany w pracy system informatyczny będzie zawierał.
Należy odnieść się do uwarunkowań prawnych związanych procesami przetwarzania danych w projektowanym systemie.
Jeśli zachodzi konieczność, należy wprowadzić i omówić model matematyczny elementów systemu na odpowiednim poziomie abstrakcji.

\subsection{Opis istniejących rozwiązań}
\subsection{Analiza istniejących rozwiązań}
\subsection{Co ulepszyc, wlasne przemyslena??}

%%%%%%%%%%%%%%%%%%%%%%%%%%%%%%%%%%%%%%%%%%%%%%%%%%%%%%%%%%%%%%%%%%%%%%%%
%%%%%%%%%%%%%%%%%%%%%%%%%%%%%%%%%%%%%%%%%%%%%%%%%%%%%%%%%%%%%%%%%%%%%%%%


\section{Analiza biznesowa}

\PRZYKLAD{W tym rozdziale przedstawiamy szczegółowy projekt systemy w notacji UML 
uwzględniający wymagania funkcjonalne opisane w \ldots. 
Do opisu relacji pomiędzy składowymi systemu wykorzystano diagramy \ldots.
Przedstawiono w pseudokodzie i omówiono algorytmy generowania \ldots.}

\subsection{Grupy użytkowników i założenia}
\subsection{SWOT}
\subsection{Analiza wymagań}
\subsubsection{Wymagania funkcjonalne}
\subsubsection{Wymagania niefunkcjonalne}
\subsubsubsection{Wykorzystane technologie i narzędzia}
\subsubsubsection{Wymagania dotyczące rozmiaru bazy danych}
\subsubsubsection{Wymagania dotyczące bezpieczeństwa systemu}

\section{Projekt systemu}
\subsection{Wykorzystane technologie}
\subsection{Przypadki użycia i scenariusze}
\subsection{Diagramy aktywności}
\subsection{Diagramy sekwencji}
\subsection{Diagramy stanu}
\subsection{Diagramy przepływu}
\subsection{Projekt bazy danych}

\section{Implementacja systemu}
\subsection{Opis kodów źródłowych}
\subsubsection{Back-end}
\subsubsection{Front-end}

\section{Instalacja i wdrożenie}

%%%%%%%%%%%%%%%%%%%%%%%%%%%%%%%%%%%%%%%%%%%%%%%%%%%%%%%%%%%%%%%%%%%%%%%%%%%%%%
%%%%%%%%%%%%%%%%%%%%%%%%%%%%%%% BIBLIOGRAFIA %%%%%%%%%%%%%%%%%%%%%%%%%%%%%%%%%
%%%%%%%%%%%%%%%%%%%%%%%%%%%%%%%%%%%%%%%%%%%%%%%%%%%%%%%%%%%%%%%%%%%%%%%%%%%%%%

\bibliographystyle{plain}
\marginpar{Znajdź w internecie porządnie zredagowane cytowania}
\bibliography{P2P}

\end{document}
